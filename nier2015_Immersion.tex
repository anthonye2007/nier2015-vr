\documentclass{acm_proc_article-sp}

\begin{document}

\title{The Future of Software Engineering with Consumer Virtual Reality}

\numberofauthors{2}
\author{
\alignauthor
Anthony Elliott\\
       \affaddr{North Carolina State University}\\
       \affaddr{Raleigh, North Carolina, USA}\\
       \email{ajellio2@ncsu.edu}
\alignauthor
Chris Parnin\\
       \affaddr{North Carolina State University}\\
       \affaddr{Raleigh, North Carolina, USA}\\
       \email{chris.parnin@ncsu.edu}
}

\maketitle
\begin{abstract}
Virtual reality can provide huge benefits to software engineering.
\end{abstract}

\section{Introduction}
Quality virtual reality (VR) is finally inexpensive enough for consumers, including software engineers, to invest in. No longer will we have to go to a physical room to be fully immersed in a virtual reality, within the next few years we will be able to simply reach over and put on our fully immersive VR sunglasses and headphones. Virtual reality is often applied only to the entertainment industry but we argue that it will greatly change software engineering as well. 

Programmers have been tied to their monitors for too long. We stare at our three dimensional images on our two dimensional displays, unable to conceive of something greater. With VR we can truly be right next to the 3D graphics that we are creating. We can see the output of our system all around us, not just displayed on our second monitor.

Software has been limited by physical world. For instance, more and larger monitors can increase programmer productivity but budget and desk size constrain this issue.
Code review can be done in a physical room, but the room cannot be easily changed to show dynamic information. White-board walls can help this physical issue, but are still not as flexible as a completely virtual environment.

Virtual reality allows developers to interact with software in a more natural environment. This immersive environment allows users to process more information at once which in turn increases comprehension.

To immerse the user in a virtual environment we used a head-mounted display (Oculus Rift) and a Leap Motion Controller for gesture recognition. These devices are inexpensive (total is less than \$500) and could feasibly be used by a number of developers within five years.

Virtual reality has been around since Ivan Sutherland's Sword of Damocles in 1968 but the hardware is finally of high enough quality with a consumer price point. This will enable widespread adoption within the decade.

[screenshot of possible setup with Rift]

\section{Related Work}
Virtual environments called CAVEs have been around since the 1990's in which a person can be in a physical room with displays covering all surfaces.  Users often use head-mounted displays with head tracking to update the displays and may use hand-held devices to enable interaction with objects in the CAVE.

Andrew Bragdon has implemented a system called Code Bubbles that enables the user to pull out methods from a file. The user is then able to move and group the methods as desired.  Code Bubbles also can display the call graph for the selected method and allows the user to open called methods.

Code Canvas added semantic zoom to the Code Bubbles tool, allowing the user to gain a better understanding of how the current section of the system fits into the system as a whole.

Chris Parnin developed a tool called NosePrints which allows users to quickly see if a code smell is relevant.

RoomAlive by Microsoft Research is similar to a CAVE but uses six Kinect sensors around the room to track the user.  This eliminates the need for wearable devices by the user but requires significant amounts of space.

\section{Applications}
\subsection{Code Review}
Immersion

[screenshot]
\subsection{Programming}
RiftSketch

[screenshot]
\subsection{Debugging}
\subsection{Visualization}
\subsection{3D Modeling}
\subsection{Metaphorical Programming}
Factory

\section{Open Research Questions}
\section{Conclusions}

\bibliographystyle{abbrv}
\bibliography{sigproc}
\end{document}
